\documentclass{article}
\usepackage[utf8]{inputenc}
\usepackage{hyperref}

\title{\textbf{SCOPING II: COMPUTATIONAL ANALYSIS}}
\author{Emily Hunt}
\date{August 2019}

\begin{document}

\maketitle

\section*{Instructions}

\textbf{First (and most important), decompose the activities that involve ‘pains’, opportunities for ‘gains’, and any solutions you proposed into small parts and/or discrete steps. Next (if possible), identify patterns in the problems you are trying to solve or the solutions you are proposing. Finally, revise the solutions you developed during BA to produce a step-by-step guide describing what you want to accomplish. To do so consider these steps:
\begin{itemize}
    \item Decomposition: Breaking down data, processes, or problems into smaller, manageable parts.
    \item Pattern Recognition: Observing patterns, trends, and regularities in data.
    \item Algorithm Design: Developing the step by step instructions for solving this and similar problems
\end{itemize}}
For this activity, I chose three of the pains/gains and relievers that I think will be the most beneficial for my further studies.

\pagebreak

\section{Pain Reliever: Reference Formatting Tool}
The pain which I was hoping to address with this tool was having to change each bibliographic reference every time a new style is required for a new submission. The tool I proposed would correctly format my references, and allow me to change them to a different style.

\subsection*{Decomposition (Drawing on Class Discussion)}
\begin{itemize}
\item The data needs to be collected (i.e. the information to make up each reference e.g. the year of publication, author, title etc.).
\item Each part of the data needs to be labelled (e.g. which part is the year, which part is the journal name etc.).
\item The output needs to be defined for each different referencing style i.e. what order does the data go in, where do the commas go, which text is in italics.
\item The location of the output needs to be determined (e.g.does it go straight into the document) and its formatting (e.g. are the references alphabetised?)
\end{itemize}

\subsection*{What I Want to Accomplish}
Revising what I did last week, I would like a tool that I can . . .
\begin{enumerate}
\item Enter the information of sources I have used.
\item Select a referencing style from a number of entered templates.
\item Export the new references, alphabetised, into the file in which I am writing my thesis. 
\end{enumerate}
I would also be beneficial if the tool could replace in-text references when the style is changed, but I am unsure how/could this could be achieved (would I have to create a new data set with each individual in-text reference, how would this be linked to the document the thesis is being written in?).

\pagebreak

\section{Gain Creator: Film Note Taking Tool}
The gain I was hoping to create with this tool was connecting notes I take on a film to the film itself that is more efficient than taking note of a time stamp and scrubbing through the film. The tool I proposed would link notes to a specific frame/section of the film. I also said it would be good if I could designate specific types of notes (e.g. notes of themes, colours) that I could be able to search for.

\subsection*{Decomposition}
\begin{itemize}
\item The input needs to be sourced (i.e. the film to be annotated) and it needs to be established whether it is required to be in a specific format.
\item The format and storage location of the notes produced needs to be determined (e.g. will the notes be stored in an excel spreadsheet).
\item If the notes are stored independently from the film file, a link needs to be created between these and the film (e.g. will a time stamp with the note take you to that specific part of the film?). A way in which these can appear while viewing the film also needs to be worked out.
\item The method of note entry needs to be designed (how will I be able to write these by watching the film).
\item It needs to be determined whether a note will link to a single frame, or whether a segment of the film will be able to be noted.
\item The specific labels for the content of the notes need to be decided. The way in which these labels will function and be able to be searched for needs to be resolved.
\end{itemize}

\subsection*{What I Want to Accomplish}
Revising what I did last week, I would like a tool that I can . . .
\begin{enumerate}
\item Write notes with while watching a film, and will link those notes to the specific frame/segment of the film.
\item View previously written notes while watching a film.
\item Read the notes independently from the film, select a note, and be taken to that point of the film.
\item Search and sort notes by various labels (e.g. notes related to cinematography, colours, framing etc.)
\end{enumerate}

\pagebreak

\section{Gain Creator: Film Colour Analysis Tool}
The gain I was hoping to create with this tool was being able to quantitatively measure the colours of a film. The tool I proposed would analyse the colour composition of an entire film. I also said it would be great if this tool allowed me to compare the colour data of two different films.

\subsection*{Decomposition}
\begin{itemize}
\item The input needs to be sourced (i.e. the film) and it needs to be established whether it is required to be in a specific format.
\item The specific way in which the colour composition of the film will be analysed needs to be determined e.g. Will each individual pixel be analysed or groups of pixels? Will each individual colour be noted or broader colour groups? Will every frame be analysed or one in every 72 frames (3 seconds)?
\item The output needs to be defined, and this can be represented (e.g. various types of graphs)
\item Additionally a way in which the output of two films could be compared would be beneficial.
\end{itemize}

\subsection*{What I Want to Accomplish}
Revising what I did last week, I would like a tool that can . . .
\begin{enumerate}
\item Analyse the colours present within a film and the frequency they appear.
\item Format this data into easily readable graphs.
\item Compare the colour data of two films.
\end{enumerate}

In this exercise I was unable to find patterns in each of my pain relievers/gain creators.\\
For future exercises, this resource may be very useful as it has tools for both of my gain creators: \url{http://dhresourcesforprojectbuilding.pbworks.com/w/page/69244319/Digital%20Humanities%20Tools#tools-video-analysis}
\\Go to the section of the page dedicated to 'Video and Film Analysis'

\end{document}
