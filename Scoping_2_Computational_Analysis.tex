\documentclass{article}
\usepackage[utf8]{inputenc}
\usepackage{hyperref}

\title{\textbf{SCOPING II: COMPUTATIONAL ANALYSIS}}
\author{Emily Hunt}
\date{August 2019}

\begin{document}

\maketitle

\section*{Gain Creator: Film Colour Analysis Tool}
The gain I was hoping to create with this tool was being able to quantitatively measure the frequency at which different colours appear in the frames of a film. The tool I proposed would analyse the colour composition of an entire film. I also said it would be great if this tool allowed me to compare the colour data of two different films.

\subsection*{Decomposition: Breaking down data, processes, or problems into smaller, manageable parts}
\begin{itemize}
\item The input needs to be sourced (i.e. the film) and it needs to be established what format this needs to be in.
\item The specific metrics of the analysis needs to be determined e.g. Will each individual pixel be analysed or groups of pixels? Will each individual colour be noted or broader colour groups? Will every frame be analysed or one in every 72 frames (3 seconds)?
\item The form of the output needs to be defined, and how this can be represented (e.g. various types of graphs).
\item A representation of the data needs to be designed that will allow the colour composition of two films to be juxtaposed (e.g. double column graph).
\end{itemize}

\subsection*{Patterns: Observing patterns, trends, and regularities in data}
\begin{itemize}
    \item Patterns may be present in the way the colours are analysed e.g. every 72 frames
\end{itemize}

\subsection*{What I Want to Accomplish (Algorithm Design: Developing the step by step instructions for solving this and similar problems)}
Revising what I did last week, I would like a tool that can . . .
\begin{enumerate}
\item Analyse the colours present within a film and the frequency they appear.
\item Format this the output of this analysis into easily readable graphs.
\item Combine the data output of two films into a single graph.
\end{enumerate}

\begin{flushleft}
For future exercises, the below link may be very useful as it has a tool for this specific gain creator, and for film note taking.Go to the section of the page dedicated to 'Video and Film Analysis'.\\
\url{http://dhresourcesforprojectbuilding.pbworks.com/w/page/69244319/Digital%20Humanities%20Tools#tools-video-analysis}
\end{flushleft}

\end{document}
